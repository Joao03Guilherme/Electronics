\begin{gnuplot}[terminal=cairolatex, scale=1]
R10 = 1.2e3        # R10 = 1.2 * 10^3 ohms
R8  = 8.2e3        # R8  = 8.2 * 10^3 ohms
R9  = 330e3        # R9  = 330 * 10^3 ohms
C4  = 10e-9        # C4  = 10 * 10^-9 farads
C3  = 8.2e-9       # C3  = 8.2 * 10^-9 farads

i = sqrt(-1)

# Define the function H(x)
H(x) = -1 / ( R8 * ( (1 / (R9 * (i * 2*pi*x) * C4)) * ( (i * 2*pi*x)*(C3 + C4) + 1/R8 + 1/R10 ) + (i * 2*pi*x * C3 ) ) )

# Define the magnitude function
f(x) = 20*log10( abs( H(x) ) )

# Define the phase function in degrees, adjusted to range from -90 to -270 degrees
phi(x) = - arg( H(x) ) * (180 / pi ) - 180

# Set up the plot for magnitude and phase response
set title "Theoretical bode plot and magnitude experimental points
set xlabel "Frequency (Hz)"
set ylabel "Gain (dB)"
set y2label "Phase (degrees)"
set xrange [1:100000]
set logscale x
set grid
set key top left
set key at 20, 27 spacing 1.4 font ",3"
set samples 1000

# Configure y-tics for both left and right axes
set ytics nomirror
set y2tics
set y2range [-400:0]

# Define the data directly in the script
$Data << EOD
6     -36.27789627
20    -24.90539774
63    -15.92740937
250   -3.775266765
500   3.950716005
600   6.782034575
628   7.576905455
700   9.931351271
800   13.63218029
900   18.88142427
950   22.70516838
970   24.17265772
1000  24.96232479
1100  20.87482441
1300  13.56766947
2000  5
7000 -8
26000 -20
EOD

# Plot the theoretical response and experimental points
plot f(x) title "Magnitude" lw 2, \
     phi(x) title "Phase" with lines dashtype 2 lw 2 lc rgb "blue" axes x1y2, \
     $Data using 1:2 with points pt 2 ps 1 lc rgb "red" title "Experimental Points"
\end{gnuplot}
