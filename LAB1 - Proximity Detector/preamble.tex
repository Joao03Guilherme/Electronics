%%%%%%%%%%%%%%%%%%%%%%%%%%%%%%%%%%%%%%%%%
% Journal Article
% LaTeX Template
% Version 1.4 (15/5/16)
%
% This template has been downloaded from:
% http://www.LaTeXTemplates.com
%
% Original author:
% Frits Wenneker (http://www.howtotex.com) with extensive modifications by
% Vel (vel@LaTeXTemplates.com)
%
% License:
% CC BY-NC-SA 3.0 (http://creativecommons.org/licenses/by-nc-sa/3.0/)
%
%%%%%%%%%%%%%%%%%%%%%%%%%%%%%%%%%%%%%%%%%

%----------------------------------------------------------------------------------------
%	PACKAGES AND OTHER DOCUMENT CONFIGURATIONS
%----------------------------------------------------------------------------------------

\usepackage{blindtext} % Package to generate dummy text throughout this template 

\usepackage[sc]{mathpazo} % Use the Palatino font
\usepackage[T1]{fontenc} % Use 8-bit encoding that has 256 glyphs
\linespread{1.05} % Line spacing - Palatino needs more space between lines
\usepackage{microtype} % Slightly tweak font spacing for aesthetics
\usepackage{float} % Force position of objects HERE

\usepackage[hang, small,labelfont=bf,up,textfont=it,up]{caption} % Custom captions under/above floats in tables or figures
\usepackage{booktabs} % Horizontal rules in tables
\usepackage{array}	% custom columns

\usepackage{lettrine} % The lettrine is the first enlarged letter at the beginning of the text

\usepackage{makecell}
\usepackage{enumitem} % Customized lists
\setlist[itemize]{noitemsep} % Make itemize lists more compact

\usepackage{abstract} % Allows abstract customization
\renewcommand{\abstractnamefont}{\normalfont\bfseries} % Set the "Abstract" text to bold
\renewcommand{\abstracttextfont}{\normalfont\small\itshape} % Set the abstract itself to small italic text

\usepackage{titlesec} % Allows customization of titles
\renewcommand\thesection{\Roman{section}} % Roman numerals for the sections
\renewcommand\thesubsection{\roman{subsection}} % roman numerals for subsections
\titleformat{\section}[block]{\large\scshape\centering}{\thesection.}{1em}{} % Change the look of the section titles
\titleformat{\subsection}[block]{\large}{\thesubsection.}{1em}{} % Change the look of the section titles

\usepackage{fancyhdr} % Headers and footers
%\pagestyle{fancy} % All pages have headers and footers
%\fancyfoot[RO,LE]{\thepage} % Custom footer text

\usepackage{titling} % Customizing the title section

\usepackage{hyperref} % For hyperlinks in the PDF
\usepackage{IEEEtrantools}
 \usepackage{mathtools}
\usepackage{amssymb} 

\usepackage{amsmath}
\usepackage{graphicx}
\usepackage{xfrac}
\usepackage{subcaption}
\usepackage{physics}

\usepackage{tikz}
\usepackage{environ}
\usepackage[export]{adjustbox}

\usepackage{csquotes}
\usepackage{multirow}
\usepackage{siunitx}
\sisetup{uncertainty-mode = separate}
\usepackage{isotope}
\usetikzlibrary{positioning,fit,calc}
\tikzset{block/.style={draw,thick,text width=2cm,minimum height=1cm,align=center},
	line/.style={-latex}
}

\makeatletter
\newsavebox{\measure@tikzpicture}
\NewEnviron{scaletikzpicturetowidth}[1]{%
  \def\tikz@width{#1}%
  \def\tikzscale{1}\begin{lrbox}{\measure@tikzpicture}%
  \BODY
  \end{lrbox}%
  \pgfmathparse{#1/\wd\measure@tikzpicture}%
  \edef\tikzscale{\pgfmathresult}%
  \BODY
}
\makeatother

%----------------------------------------------------------------------------------------
%	TITLE SECTION
%----------------------------------------------------------------------------------------

\setlength{\droptitle}{-4\baselineskip} % Move the title up

\pretitle{\begin{center}\Huge\bfseries} % Article title formatting
\posttitle{\end{center}} % Article title closing formatting

%================================================================================
%	MATHEMATICAL ABBREVIATIONS
%================================================================================
\renewcommand{\inf}{\infty}		% infinity
\newcommand{\I}{{i\mkern1mu}}	% sqrt(-1)
\newcommand{\ee}{\mathrm{e}}	% euler constant
\renewcommand{\exp}[1]{\mathrm{e}^{\left\{#1\right\}}}	% exponential function
\newcommand{\wt}[1]{\widetilde{#1}}
\newcommand{\ft}{\widetilde{\phi}}	% wide til phi
\newcommand{\qed}{\hfill\blacksquare}
\newcommand{\pdt}[1][]{\partial_{t}^{#1}}		% partial time derivative
\newcommand{\pdx}[1][]{\partial_{x}^{#1}}		% partial time derivative

%\newcommand{\el}{\mathrm{e^{-}}		% electron
%\newcommand{\ep}{\mathrm{e^{+}}}		% positron
%================================================================================
%	PLASMA ABBREVIATIONS
%================================================================================
\newcommand{\qT}{q_{_{T}}}			% test charge
\newcommand{\qs}{q_{_{s}}}			% species charge
\newcommand{\qo}{q_{_{0}}}			% charge
\newcommand{\vvo}{\vb{v}_{_0}}			% species velocity
\newcommand{\vvs}{\vb{v}_{_s}}			% species velocity
\newcommand{\vve}{\vb{v}_{_e}}			% electron velocity
\newcommand{\vvi}{\vb{v}_{_i}}			% ion velocity
\newcommand{\no}{n_{_{0}}}			% density
\newcommand{\ns}[1][]{n_{_{s#1}}}	% density: species [order]
\renewcommand{\ne}[1][]{n_{_{e#1}}}	% density: electron [order]
\renewcommand{\ni}[1][]{n_{_{i#1}}}	% density: ion [order]
\newcommand{\Ts}{T_{_{s}}}			% temperature: species
\newcommand{\Te}{T_{_{e}}}			% temperature: electron
\newcommand{\Ti}{T_{_{i}}}			% temperature: ion 
\newcommand{\ms}{m_{_{s}}}			% mass: species 
\newcommand{\me}{m_{_{e}}}			% mass: electron 
\newcommand{\mi}{m_{_{i}}}			% mass: ion
\newcommand{\evac}{\epsilon_{_{0}}}	% vacuum permeability
\newcommand{\lD}[1][]{\lambda_{_{D{#1}}}}	% [species] Debye's length
\newcommand{\kB}{k_{_B}}					% Boltzmann's constant
\newcommand{\dB}[1][]{\frac{e\phi}{\kB T_{_{#1}}}}	% Boltzmann distribution
%================================================================================
%	QUANTUM MECHANICS ABBREVIATIONS
%================================================================================
\newcommand{\ac}{a^{\dagger}}			% creation operator
\newcommand{\sa}{\ket{\alpha}}			% state 
\newcommand{\sacc}{\bra{\alpha}}			% state complex conjugate
\newcommand{\cO}{\mathcal{O}}			% Order

%\csname endofdump\endcsname

\usepackage[english]{babel} % Language hyphenation and typographical rules
\usepackage{derivative}		% derivative  

\newcommand{\pmseparator}{\hspace{0.2mm}\ensuremath{\pm}\hspace{0.2mm}}
\newcolumntype{+}{@{\pmseparator}}
%\endofdump
\usepackage[shell]{gnuplottex}